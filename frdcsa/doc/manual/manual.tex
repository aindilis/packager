\FOT{2}\Node%
{}\Node%
{\def\Element%
{0}}\Node%
{\def\Element%
{1}}
    Packager Codebase
  \endNode{}\Node%
{\def\Element%
{2}}\Node%
{\def\Element%
{3}}\Node%
{\def\Element%
{4}}Andrew\endNode{}\Node%
{\def\Element%
{5}}Dougherty\endNode{}\endNode{}\endNode{}\Node%
{\def\Element%
{6}}\endNode{}\Node%
{\def\Element%
{7}}\Node%
{\def\Element%
{8}}What is Packager\endNode{}\Node%
{\def\Label%
{overview}}\Node%
{\def\Element%
{10}}
	The Problem
      \endNode{}\Node%
{\def\Element%
{11}}
	On the Need to Package Software

	There are many software systems out there that provide what
	has been termed islands of ability, but these islands must be
	brought together to create a functioning system.  From a
	practical point of view, this entails creating software
	packages for these.  However, there are many good systems for
	which there are no packages made.

	This is an obvious inefficiency, because the cost in time and
	expertise for packaging	software is definitely nonzero, so
	rather than requiring O(N*M) effort to generate N packages for
	M people, we simply require O(N) effort.

	How to Package Software

	There are not in practice enough people it seems to package
	all the existing packages, even in terms of creating
	low-\/quality packages.
      \endNode{}\endNode{}\Node%
{\def\Element%
{12}}\Node%
{\def\Element%
{13}}
	Towards a Solution
      \endNode{}\Node%
{\def\Element%
{14}}
	Therefore, the ability to automatically generate packages is
	desired.  This is what Packager solves, it can automatically
	or semi-\/automatically package software.
	
	For practical reasons, we have chosen to go with Debian
	packages as a starting point, and then either improve the
	Packager application or simply translate the packages
	using a tool like Alien or an improvement of it.

	Packager mainly implements a  system to create Debian packages
	out of arbitrary upstream codebases.
      \endNode{}\endNode{}\endNode{}\Node%
{\def\Element%
{15}}\Node%
{\def\Element%
{16}}
      Packaging Techniques
    \endNode{}\Node%
{\def\Element%
{17}}\Node%
{\def\Element%
{18}}
	The Packaging Process
      \endNode{}\Node%
{\def\Element%
{19}}
	The packaging process is semi-\/supervised, of course, since the
	system  does  not begin  with  all  of  the capabilities.   Of
	course,  given the  domain, the  architecture itself  is under
	constant elaboration  and the development  of the architecture
	is considered part of its learning.
      \endNode{}\Node%
{\def\Element%
{20}}
	Importantly, the packages which are produced by early versions
	of Packager can later  be used to facilitate capabilities.  An
	example  of this  might be  that a  system for  annotating and
	extracting  text files  could  be packaged  with the  current,
	hand-\/written perl implementation of Packager, and then used to
	annotate  files  for  extraction  to increase  the  extent  of
	automation.
      \endNode{}\Node%
{\def\Element%
{21}}
	In  addition to  "learning"  via programming,  there are  real
	learning systems employed.  The  first is that any action that
	the user  takes in excess of  Packager in order  to complete a
	package must  be formalizeable  as an action.   By documenting
	the various incompletenesses of Packager, it will be easier to
	add  features or  train existing  learning methods  to perform
	certain tasks.
      \endNode{}\Node%
{\def\Element%
{22}}
	The basic  process is currently Make like.   There are several
	targets which  relate to stages  of packaging.  Most  of these
	steps are  performed by existing  packaging assistants, almost
	all of  which are clearly specified in  the Debian Maintainers
	Manual.  Incidentally,  we intend to formalize  the process of
	taking  documents  like the  Debian  Developers' Policies  and
	mapping the  logic documented therein  into code, so  that, as
	best practices and online documentation are improved, updating
	the  code to reflect  this is  an entirely  auditable process.
	Obviously, the  notion of process is very  important here, but
	we  have  yet  to  find  an appropriate  system  for  robustly
	modelling processes (as opposed to simply hard coding them via
	software).  This remains a  major need.  We have explored many
	techniques but none have been employed successfully, mainly on
	account  of  our  own  ignorance  and lack  of  resources  and
	assistance.  (We have, for instance, looked at AIAI's I-\/DE and
	I-\/X, Flowdesigner,  ArgoUML, AcmeStudio, OntoEdit,  etc.  Many
	times we have  not succeeding in getting the  software to even
	run.  Although this could be  viewed as ineptitude, it is more
	appropriate  and accurate  to consider  it lack  of resources.
	For  instance, during  the  aforementioned testing  I had  not
	eaten in  6 days,  save 2  cans of red  kidney beans  and some
	hotdogs  one day.  Secondly,  this may  be attributable  to my
	ignorance of Software  Engineering and Architecture practices.
	However,  I   feel  that  such   practices  are  prohibitively
	difficult to  learn, and seek  assistance to do so.   The fact
	that there are not existing  tools which more or less complete
	the  process  indicates a  certain  lack  of  rigour in  these
	approaches.)
      \endNode{}\Node%
{\def\Element%
{23}}
	So, as the software is  packaged, more cases for more software
	is elaborated.  A fundamental  problem is that the information
	does not come  in any standard form, if  at all, and therefore
	has  to be  estimated.   One instance  of  this is  dependency
	information.  There are hybrid methods for accomplishing this.
	First is, every  software package that is added,  the files it
	contains are  mapped to the desired  metadata.  Therefore, the
	system  should be able  to look  in appropriate  places, using
	appropriate features, to extract the desired information.  For
	instance, the  system should know  to look in readmes,  and to
	incorporate the names of known systems to approximately filter
	the readme  file for  dependency information.  In  some cases,
	not  all the  software  is necessary,  and  so the  annotation
	language,  which  is   an  ontology,  must  include  relations
	specifying this.  There are natural kinds of dependencies with
	Debian (suggets, depends,  recommends, conflicts), etc.  Using
	other tools like pbuilder and sbuild, it should be possible to
	do  some   measure  of  automatic   testing  of  dependencies.
	Integrating  both modes  should  give a  head start.   Lastly,
	data-\/mining  of dependencies  versus  various features  should
	allow prediction  of what libraries are needed.   All of these
	techniques  will  substantially  reduce,  and  in  some  cases
	eliminate completely,  the effort of  the packaging supervisor
	to manually  locate software.  Of  course it is also  the case
	that the  system will be invoked on  each dependency, thereby,
	automatically   finding   and   packaging   (where   possible)
	dependencies.    This  information   will  be   used   by  the
	Machiavelli system in  authoring persuasive letters to authors
	of software, based on further confidence building tests -\/ such
	as applying question answering software to questions like "why
	isn't system X licensed open source", and user given approval,
	to write  letters to  the authors in  the hopes  of persuading
	them  to  relicense  their  software, as  well  as  key-\/player
	network analysis measures to focus efforts on important cases.
	Incidentally,  the  persuasive   letter  authoring  system  is
	already completed.
      \endNode{}\Node%
{\def\Element%
{24}}
	In conclusion, there are variety of special methods which must
	be  invoked  to  extract  information  necessary  to  creating
	packages.   The system  attempts  to monitor  the process  and
	activities of packagers in  order to identify commonalties and
	areas for improvement.  The system evolves both through manual
	coding, and automatic learning.  The system interacts with the
	user, proceeding as far as  possible before asking the user to
	complete  some task.   I  forgot to  mention  that a  planning
	process  is  necessary,  and  should  be part  of  the  larger
	planning  system,  and  therefore  able to  exchange  tactical
	information  to improve the  overall quality  of automatically
	maintained development plans.
      \endNode{}\endNode{}\Node%
{\def\Element%
{25}}\Node%
{\def\Element%
{26}}
	Examples of Architectural Components
      \endNode{}\Node%
{\def\Element%
{27}}
	So  far, here  are some  of  the tools  which have  been
	incorporated to give Packager its functionality.
      \endNode{}\Node%
{\def\Element%
{28}}
	Mainly it  is perl code.  We  have done some  UML modelling of
	the process, mainly the capabilities management.
      \endNode{}\Node%
{\def\Element%
{29}}
	Without a doubt, the next  few steps will involve semantic web
	technologies, mainly ontology tools.  The package state should
	be represented  ontologically.  Therefore, modules  will query
	the ontology system in order to make decisions.
      \endNode{}\Node%
{\def\Element%
{30}}
	The process definition is important and we have not settled on
	what  this will  be.  Neither  have we  found  the information
	extraction system,  that is  capable of handling  diverse file
	types (such  as .deb and tgz,  pdf, etc.)  Wow,  that would be
	powerful,  wouldn't it?!  I'll  have to  look for  that.  Now,
	here is an example, one  would simply, using the current state
	of  our technology,  do  this :
      \endNode{}\Node%
{\def\Element%
{31}}
	radar  search  -\/c  "information  extraction  system,  that  is
	capable of handling diverse file  types (such as .deb and tgz,
	pdf, etc.)"
      \endNode{}\Node%
{\def\Element%
{32}}
	This command  would then begin a  capabilites matching search,
	by looking at the capabilities already listed in the ontology,
	and by initiating a new search with this specific criteria.
      \endNode{}\Node%
{\def\Element%
{33}}
	Lastly, there are numerous  Debian specific tools which do all
	the real work at  this point, including dput, debmake, dh\char95{}make
	dpkg-\/buildpackage,     cvs-\/buildpackage,    lintian,    linda,
	mini-\/dinstall, etc.
      \endNode{}\endNode{}\Node%
{\def\Element%
{34}}\Node%
{\def\Element%
{35}}
	Policy
      \endNode{}\Node%
{\def\Element%
{36}}
	The  policy of  Packager is  to collect  and  use applications
	according   to   their   licenses  and/or   specific   written
	agreements.  A sophisticated license management system will be
	found-\/or-\/created to ensure that all software, indeed all data,
	is used in  accordance with its authors wishes.   We intend to
	formalize  the contents of  licenses so  that we  can formally
	prove  the  correctness of  the  software  use.  This  license
	extraction is already done in  a minor capacity, but seeing as
	this  is   critical,  this  capability   will  be  extensively
	developed.
      \endNode{}\Node%
{\def\Element%
{37}}
	Due to  bounded resources, we  are currently employing  a wait
	and  see  approach.  Our  planning  technology is  essentially
	unused, i.e. there is not even basic goal dependency analysis.
	This is a  major problem.  If we got this  done, we would like
	to then  focus on other  things, like adequately  planning and
	developing  the essential  software.   So, right  now, we  are
	simply  going with  an  approach to  create essential  systems
	mainly by  locating, installing  and testing them,  since this
	still provides greater code capability.
      \endNode{}\endNode{}\endNode{}\endNode{}\endNode{}\endFOT{}